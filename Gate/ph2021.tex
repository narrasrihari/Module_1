\documentclass[12pt,-letter paper]{article}
\usepackage{siunitx}
\usepackage{setspace}
\usepackage{gensymb}
\usepackage{xcolor}
\usepackage{caption}
%\usepackage{subcaption}
\doublespacing
\singlespacing
\usepackage[none]{hyphenat}
\usepackage{amssymb}
\usepackage{relsize}
\usepackage[cmex10]{amsmath}
\usepackage{mathtools}
\usepackage{amsmath}
\usepackage{commath}
\usepackage{amsthm}
\interdisplaylinepenalty=2500
%\savesymbol{iint}
\usepackage{txfonts}
%\restoresymbol{TXF}{iint}
\usepackage{wasysym}
\usepackage{amsthm}
\usepackage{mathrsfs}
\usepackage{txfonts}
\let\vec\mathbf{}
\usepackage{stfloats}
\usepackage{float}
\usepackage{hyperref}
\usepackage{cite}
\usepackage{cases}
\usepackage{subfig}
%\usepackage{xtab}
\usepackage{longtable}
\usepackage{multirow}
%\usepackage{algorithm}
\usepackage{amssymb}
%\usepackage{algpseudocode}
\usepackage{enumitem}
\usepackage{mathtools}
%\usepackage{eenrc}
%\usepackage[framemethod=tikz]{mdframed}
\usepackage{listings}
%\usepackage{listings}
\usepackage[latin1]{inputenc}
%%\usepackage{color}{   
%%\usepackage{lscape}
\usepackage{textcomp}
\usepackage{titling}
\usepackage{hyperref}
%\usepackage{fulbigskip}   
\usepackage{tikz}
\usepackage{graphicx}
\lstset{
  frame=single,
  breaklines=true
}
\let\vec\mathbf{}
\usepackage{enumitem}
\usepackage{graphicx}
\usepackage{siunitx}
\let\vec\mathbf{}
\usepackage{enumitem}
\usepackage{graphicx}
\usepackage{enumitem}
\usepackage{tfrupee}
\usepackage{amsmath}
\usepackage{amssymb}
\usepackage{mwe} % for blindtext and example-image-a in example
\usepackage{wrapfig}
\graphicspath{{figs/}}
\providecommand{\mydet}[1]{\ensuremath{\begin{vmatrix}#1\end{vmatrix}}}
\providecommand{\myvec}[1]{\ensuremath{\begin{bmatrix}#1\end{bmatrix}}}
\providecommand{\cbrak}[1]{\ensuremath{\left\{#1\right\}}}
\providecommand{\brak}[1]{\ensuremath{\left(#1\right)}}
\providecommand{\norm}[1]{\left\lVert#1\right\rVert}
\providecommand{\abs}[1]{\left\vert#1\right\vert}
\usepackage{circuitikz}

\title{Assignment}
\date{\today}
\begin{document}
\maketitle
\section*{Gate question}
             \begin{circuitikz}
              
                  \draw (2,5) node[not port,scale=2] (not) {};
                  \draw (not.in) -- ++(-0.5,0);
                  \draw (not.out) -- ++(0.5,0) ;
                  
                
                
                \draw (7.5,3) node[or port,scale=2] (orgate) {};
                  \draw (orgate.in 1) -- ++(-0.5,0) ;
                  \draw (orgate.in 2) -- ++(-0.5,0);
                  \draw (orgate.out) -- ++(0.5,0) ;
                
                 \draw (2,1) node[not port,scale=2] (notgate) {};
                  \draw (notgate.in) -- ++(-0.5,0) ;
                  \draw (notgate.out) -- ++(0.5,0) ;
                
                \draw (notgate.out) -- ([xshift=0.5cm]notgate.out) |-(orgate.in 2);
                  
                \draw (not.out) -- ([xshift=0.5cm]not.out) |-(orgate.in 1);
                  
            \end{circuitikz}
    \begin{enumerate}
      \item  The above combination of logic gate represent the operation
       \begin{enumerate}
       \item OR
       \item NAND
       \item AND
       \item NOR
   \end{enumerate}
   \end{enumerate}
   
\end{document}
